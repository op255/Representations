\documentclass{article}
\usepackage[utf8]{inputenc}
\usepackage{amssymb,amsmath}
\usepackage[usenames, dvipsnames]{color}
\usepackage{amsthm}
\textheight=24cm % высота текста
\textwidth=15cm % ширина текста
\oddsidemargin=1.0cm % отступ от левого края
\topmargin=-1.5cm % отступ от верхнего края
\parindent=24pt % абзацный отступ
\parskip=0pt % интервал между абзацами
\tolerance=2000 % терпимость к "жидким" строкам
\flushbottom % выравнивание высоты страниц
\newtheorem{theorem}{Theorem}
\newtheorem{definition}{Definition}
\newtheore{lemma}{Lemma}

\title{Representations of flat virtual braids which do not preserve the forbidden relations}
\author{B.~Chuzhinov \and I.~Emelianenkov \and  M.~Ivanov \and E.~Markhinina  \and S.~Panov \and  T.~Plevako \and N.~Singh \and S.~Vasyutkin \and V.~Yakhin \and V.~Bardakov \and A.~Vesnin \and T.~Nasybullov} 
\date{July 2020}
\usepackage{natbib}
\usepackage{graphicx}

\begin{document}
\maketitle
\begin{abstract}
 Short formulation of all our results.
 
 
 ~\\
 \textit{Keywords:}
 
 ~\\
 \textit{Mathematics Subject Classification:} 
\end{abstract}

\section{Introduction}
Motivation here: knot theory, flat virtual knots, braids, flat virtual braids, representations of braid groups, question of Bardakov from \cite{problems}

Everyone comes across knots. For example, everybody knows about shoelace knots, neck tie knots, knots in logos. 

To understand what is the mathematical knot imagine one experiment. Take a long rope, tangle it and glue the ends. This is a mathematical knot. In mathematical term mathematical knot is  a closed curve in the Euclidean space $R^3$. \\

A link is two or more knots which may or may not be entangled.

Two knots/links are called equivalent if one can be deformed into other by wiggling the knot in $R^3$. The knots are allowed to stretch, shrink and twist but not allowed to cross themselves.

One of the basic problems of the knot theory is the problem of determining by two given links equivalent they are or not.\\

Leaning braid helps us to study knots. Braids on $n$ threads are a set of  $n$ threads entangled with each other which connect $n$ points on one straight line with $n$ points on a parallel straight line.(image)

Braids on $n$ strands can be multiplied with each other. The multiplication $\alpha\beta$ of the braid $\alpha$ on $n$ strands with the braid $\beta$ on $n$ strands is the braid $\alpha$ obtained from the braid $\beta$  by adding braid $\beta$ to it from below. (image)

So the set of all braids on $n$ strands with the multiplication operation defined in this way forms a group, which is denoted by the symbol $B_n$ and is called a braid group on $n$ strands.

In mathematical term, a braid group $B_n$ is a group with generators $\sigma _1,\sigma _2,\dots,\sigma _{n-1}$ and defining relations 

\begin{align*}
\sigma _i\sigma _{i+1}\sigma _i &= \sigma _{i+1}\sigma _i\sigma _{i+1}&&i = 1, 2, \dots. n-2\\
\sigma _i\sigma _j &= \sigma _j\sigma _i&&|i-j|\geqslant   2\\
\end{align*}


Braids are objects closely related to knots. If $\alpha$ is a braid on $n$ strands, then to connect the first upper point to the first bottom point, the second upper braid point to the second bottom braid point, and so on we get some link $\check  \alpha$ (image).

For every link $K$ exists some braid $\alpha$, where link $K$ and $\check  \alpha$ are equivalent.\\


The group of flat virtual braids on $n$ strands is the group $FVB_n$ with generators  $\sigma _1,\sigma _2,\dots,\sigma _{n-1}\rho_1,\rho _2,\dots,\rho _{n-1}$ and defining relations 


\begin{align}
\notag\sigma_i^2&=1 && i = 1, 2, \dots,n-1,\\
\notag\sigma_i\sigma _{i+1}\sigma _i &= \sigma _{i+1}\sigma _i\sigma _{i+1}&& i = 1, 2, \dots. n-2, \\
\notag\sigma _i\sigma _j &= \sigma _j\sigma _i && |i-j|\geqslant   2,\\
\label{def}\rho_i^2&=1 &&i = 1, 2, \dots,n-1, \\
\notag\rho_i\rho _{i+1}\rho _i &= \rho _{i+1}\rho_i\rho _{i+1} &&i = 1, 2, \dots. n-2,\\
\notag\rho _i\rho _j &= \rho _j\rho _i && |i-j|\geqslant 2,\\
\notag\sigma _i\rho _j &= \rho _j\sigma _i &&|i-j|\geqslant   2,\\
\notag\rho_i\rho _{i+1}\sigma _i &= \sigma _{i+1}\rho_i\rho _{i+1} && i = 1, 2, \dots. n-2.
\end{align}


Our aim is construct a representation $FVB_n \rightarrow {\rm Aut} (G)$, where $G$ is some "good" group, for example, a free group.
\section{Preliminaries}

We need to know a few definitions and statesments that we will use in our work.

First of all let's notice that all braid groups smoothly going to flat virtual braids what was proven by V.~Bardakov, P.~Bellingeri, C.~Damiani in \cite{BarBelDom}. 

Let's make sure, the flat virtual braids $FVB_n$ is a quotient of $VB_n$ adding relations 
$$ \sigma_i^2=1, \quad i = 1, 2, \dots,n-1 $$

It is evident that $FVB_n$ is a quotient of the free product $S_n*S_n$

Then we can consider the natural projection map $f:VB_n \rightarrow FVB_n$, and set $f(\rho_i):= \rho_i$ and $f(\sigma_i):= s_i$ for $i = 1, 2, \dots, n-1$. 

In addition to relations coming from the two copies of $S_n$, in $FVB_n$ we have mixed relations 
\begin{align*}
s_i\rho_j = \rho_js_i, &\quad |i-j| \ge 2,\\
\rho_i\rho_{i+1}s_i = s_{i+1}\rho_i\rho_{i+1}, &\quad i = 1, 2, \dots, n-2.
\end{align*}

So now we can define \textit{flat pure virtual braid group} $FVP_n$ as the kernel of the map  $FVB_n \to S_n$ defined by $s_i, \rho_i \mapsto (i, i+1)$ for $i = 1, 2, \dots, n-1$. With respect to the map $f:VB_n \to FVB_n$, we have that $f(VP_n) = FVP_n$. 

The definition follows that the group $FVB_n$ is isomorphic to $FVP_n\rtimes S_n$.

\vspace{6pt}Second of all we have the description of $FVP_n$ in case $n = 3$ proposed in \cite{BarBelDom}.
$$FVP_3 = \langle a, b, c ~|| ~ [a,c] = 1 \rangle  = \langle a, c~||~[a,c] = 1 \rangle * \langle b \rangle$$

Where generators defined by the formulas:
$$ 
\begin{cases}
a = \rho_2\sigma_2\rho_2\rho_1\sigma_1\rho_2, \\
b = \rho_2\sigma_2,\\
c = \rho_2\sigma_2\sigma_1\rho_2.\\
\end{cases}
$$

This result will be useful when we search the kernel of representation.

\vspace{6pt}In the course of our work we found many representations of $FVB_n$, but lots of them were useless. Why we will understand by example:
$$\Phi : FVB_n \to Aut(F_n+1)$$
$$
\Phi(\sigma_i):
\begin{cases}
x_i \mapsto x_{i+1},\\
x_{i+1} \mapsto x_i,\\
x_j \mapsto x_j \quad for~j \not= i, i+1 \\
y \mapsto y \\
\end{cases},\quad
\Phi(\rho_i):
\begin{cases}
x_i \mapsto yx_{i+1}y^{-1},\\
x_{i+1} \mapsto t^{-1}x_iy,\\
x_j \mapsto x_j \quad for~j \not= i, i+1 \\
y \mapsto y \\
\end{cases}
$$

It is easy to check that all defining relations are correct.

For example let's check mixed relation $\rho_{i+1}\rho_i\sigma_{i+1} = \sigma_i\rho_{i+1}\rho_i$

\begin{align*}
\Phi(\rho_{i+1}\rho_i\sigma_{i+1}):&
\begin{cases}
x_i \mapsto x_i \mapsto yx_{i+1}y^{-1} \mapsto yx_{i+2}y^{-1},\\
x_{i+1} \mapsto yx_{i+2}y^{-1} \mapsto yx_{i+2}y^{-1} \mapsto yx_{i+1}y^{-1},\\
x_{i+2} \mapsto y^{-1}x_{i+1}y \mapsto y^{-2}x_iy^2 \mapsto y^{-2}x_iy^2,\\
y \mapsto y\mapsto y \mapsto y, \\
\end{cases} \\
\Phi(\sigma_i\rho_{i+1}\rho_i):&
\begin{cases}
x_i \mapsto x_{i+1} \mapsto yx_{i+2}y^{-1} \mapsto yx_{i+2}y^{-1},\\
x_{i+1} \mapsto x_i \mapsto x_i \mapsto yx_{i+1}y^{-1},\\
x_{i+2} \mapsto x_{i+2} \mapsto y^{-1}x_{i+1}y^ \mapsto y^{-2}x_iy^2,\\
y \mapsto y\mapsto y \mapsto y, \\
\end{cases}
\end{align*}

Others by anology.


It means that $\Phi$ actually are representation for $FVB_n$. But forbidden relation $\rho_i\sigma_{i+1}\sigma_i = \sigma_{i+1}\sigma_i\rho_{i+1}$ is correct too.
\begin{align*}
\Phi(\rho_{i+1}\rho_i\sigma_{i+1}):&
\begin{cases}
x_i \mapsto yx_{i+1}y^{-1} \mapsto yx_{i+2}y^{-1} \mapsto yx_{i+2}y^{-1},\\
x_{i+1} \mapsto yx_iy^{-1} \mapsto yx_iy^{-1} \mapsto y^{-1}x_{i+1}y,\\
x_{i+2} \mapsto x_{i+2} \mapsto x_{i+2} \mapsto x_i,\\
y \mapsto y\mapsto y \mapsto y, \\
\end{cases} \\
\Phi(\sigma_i\rho_{i+1}\rho_i):&
\begin{cases}
x_i \mapsto x_i \mapsto x_{i+1}  \mapsto yx_{i+2}y^{-1},\\
x_{i+1} \mapsto x_{i+2} \mapsto x_{i+2} \mapsto y^{-1}x_{i+1}y,\\
x_{i+2} \mapsto x_{i+1} \mapsto x_i\mapsto x_i,\\
y \mapsto y\mapsto y \mapsto y, \\
\end{cases}
\end{align*}

 what leads to problems with corresponding witn knot theory. 

\vspace{6pt}And the last preliminary statesment wich were proved in \cite{BarBelDom} is that the group $FVB_n$ is linear. The proof based on facts that $FVB_n = FH_n \rtimes S_n$ and that all finitely generated Coxeter groups are linear and that finite extensions of linear groups are also linear. That realy useful fact makes us to find a linear representation of that group, what we have done.

\section{New representation}

Consider a free group $F_{2n}=\langle x_1, x_2,\ldots , x_n, y_1, y_2, \ldots , y_n \rangle$. M.~Ivanov and I.~Emelianenkov suggested the mapping $\theta_n:FVB_n \rightarrow {\rm Aut}(F_{2n})$ which acts on the generators as follows:
$$
\theta_n(\sigma_i):
\begin{cases}
x_i \mapsto x_{i+1}y_{i+1}\\
x_{i+1} \mapsto x_iy_{i+1}^{-1}\\
\end{cases}\quad
\theta_n(\rho_i):
\begin{cases}
x_i \mapsto x_{i+1}\\
x_{i+1} \mapsto x_i\\
y_i \mapsto y_{i+1}\\
y_{i+1} \mapsto y_i\\
\end{cases}
$$

The main result of this section is the following

\begin{theorem}
The mapping $\theta_n:FVB_n \rightarrow {\rm Aut}(F_{2n})$ as defined above is representation of flat virtual braids which do not preserve the forbidden relations.
\end{theorem} 

\begin{proof}
To show that $\theta_n$ is representation sufficinatly check that images of generators satisfy the relations (\ref{def}). Denote by $s_i$ and $r_i$ images of $\sigma_i$ and $\rho_i$ respectively. Agree to understand the composition $sr$ of mappings $s$ and $r$ as sequential action first $s$ then $r$.

It is not hard to see that $s_is_j=s_js_i$, $r_ir_j=r_jr_i$ and $s_ir_j=r_js_i$ for $|i-j|\ge2$. Indeed, automorphisms $s_i$ and $r_i$ do not act identically only on $x_i, x_{i+1}, y_i, y_{i+1}$. Then for $|i-j|\ge2$  $s_i$ acts identically on  $x_j, x_{j+1}, y_j, y_{j+1}$ and $s_j$ acts identically on  $x_i, x_{i+1}, y_i, y_{i+1}$. Therefore $s_is_j=s_js_i$. Similarly with the equalities $r_ir_j=r_jr_i$ and $s_ir_j=r_js_i$ for $|i-j|\ge2$.

The calculations below prove that the remaining relations of flat virtual braids' group hold, and thus it is proved that $\theta_n$ -- representation.

\begin{align*}
s_is_{i+1}s_i &:
\begin{cases}
x_i\xrightarrow{s_i} x_{i+1}y_{i+1}\xrightarrow{s_{i+1}} x_{i+2}y_{i+2}y_{i+1}\xrightarrow{s_i}x_{i+2}y_{i+2}y_{i+1}\\
x_{i+1}\xrightarrow{s_i} x_{i+1}\xrightarrow{s_{i+1}} x_i\xrightarrow{s_{i}} x_{i+1}\\
x_{i+2}\xrightarrow{s_i} x_{i+2}\xrightarrow{s_{i+1}} x_{i+1}y_{i+2^{-1}}\xrightarrow{s_i} x_iy_{i+1}^{-1}y_{i+2}^{-1}\\
\end{cases} \\
s_{i+1}s_is_{i+1} &:
\begin{cases}
x_ix\rightarrow{s_{i+1}} x_i\xrightarrow{s_i} x_{i+1}y_{i+1}\xrightarrow{s_{i+1}} x_{i+2}y_{i+2}y_{i+1}\\
x_{i+1}\xrightarrow{s_{i+1}} x_{i+2}y_{i+2}\xrightarrow{s_i} x_{i+2}y_{i+2}\xrightarrow{s_{i+1}} x_{i+1}\\
x_{i+2}\xrightarrow{s_{i+1}} x_{i+1}y_{i+2}^{-1}\xrightarrow{s_{i}} x_iy_{i+1}^{-1}y_{i+2}^{-1}\xrightarrow{s_{i+1}} x_iy_{i+1}^{-1}y_{i+2}^{-1}\\
\end{cases} 
\end{align*}

The rest is calculated similarly:

\begin{align*}
r_ir_{i+1}r_i &:
\begin{cases}
x_i\mapsto x_{i+2}\\
x_{i+1}\mapsto x_{i+1}\\
x_{i+2}\mapsto x_i\\
y_{i} \mapsto y_{i+2}\\
y_{i+1} \mapsto y_{i+1}\\
y_{i+2} \mapsto y_{i}\\
\end{cases} &\qquad
r_{i+1}r_ir_{i+1} &:
\begin{cases}
x_i\mapsto x_{i+2}\\
x_{i+1}\mapsto x_{i+1}\\
x_{i+2}\mapsto x_i\\
y_{i} \mapsto y_{i+2}\\
y_{i+1} \mapsto y_{i+1}\\
y_{i+2} \mapsto y_{i}\\
\end{cases} \\ 
r_{i+1}r_is_{i+1} &:
\begin{cases}
x_i\mapsto x_{i+2}y_{i+2}\\
x_{i+1}\mapsto x_{i+1}y_{i+2}^{-1}\\
x_{i+2}\mapsto x_iy_{i+1}^{-1}\\
y_{i} \mapsto y_{i+1}\\
y_{i+1} \mapsto y_{i+2}\\
y_{i+2} \mapsto y_{i}\\
\end{cases} &\qquad
s_ir_{i+1}r_i &:
\begin{cases}
x_i\mapsto x_{i+2}y_{i+2}\\
x_{i+1}\mapsto x_{i+1}y_{i+2}^{-1}\\
x_{i+2}\mapsto x_iy_{i+1}^{-1}\\
y_{i}\mapsto y_{i+1}\\
y_{i+1}\mapsto y_{i+2}\\
y_{i+2}\mapsto y_{i}\\
\end{cases} 
\end{align*}

Finally, $r_is_{i+1}s_i(x_{i+2})=x_iy_{i+1}^{-1}y_{i+2}^{-1}$ but $s_{i+1}s_ir_{i+1}(x_{i+2})=x_iy_{i+2}^{-1}y_{i+1}^{-1}$. Also $r_{i+1}s_is_{i+1}(x_i)=x_{i+2}y_{i+2}y_{i+1}$ but $s_is_{i+1}r_i(x_i)=x_{i+2}y_{i+2}y_i$. Thus $\theta_n$ do not preserve the forbidden relations.    
\end{proof}

Remark that it solves the problem of V.~Bardakov from \cite{problems}
\section{What is the kernel}
Introduce the element from the kernel which we already have

Prove that this element is not trivial (use the discription of $FVP_3$ and express this element in terms of $a,b,c$)

Introduce the group $H$

The mapping $\theta$ acts on $a, b$ and $c$ in the following way:
$$
\theta(a) = \alpha : 
\begin{cases}
	x_1 \rightarrow x_1 y_3^{-1}\\
	x_2 \rightarrow x_2 y_1^{-1}\\
	x_3 \rightarrow x_3 y_3 y_1\\
	y_1 \rightarrow y_3\\
	y_2 \rightarrow y_1\\
	y_3 \rightarrow y_2
\end{cases},
\theta(b) = \beta :
\begin{cases}
	x_1 \rightarrow x_1\\
	x_2 \rightarrow x_2 y_3^{-1}\\
	x_3 \rightarrow x_3 y_3\\
	y_1 \rightarrow y_1\\
	y_2 \rightarrow y_3\\
	y_3 \rightarrow y_2
\end{cases},
\theta(c) = \gamma :
\begin{cases}
	x_1 \rightarrow x_1 y_1\\
	x_2 \rightarrow x_2 y_1^{-1} y_3^{-1}\\
	x_3 \rightarrow x_3 y_3\\
	y_1 \rightarrow y_2\\
	y_2 \rightarrow y_3\\
	y_3 \rightarrow y_1
\end{cases}.
$$

Proposition ${\rm ker}\cap H=1$. 

Proof. An arbitrary element in $\theta(H)$ has the form $\alpha^r \gamma^s = (\alpha \gamma)^r \gamma^{s-r}$ for some $r,s \in \mathbb{Z}$. It is easy to see that
$$
(\alpha \gamma)^r :
\begin{cases}
	x_1 \rightarrow x_1\\
	x_2 \rightarrow x_2 (y_1 y_3 y_1)^{-r}\\
	x_3 \rightarrow x_3 (y_3 y_1 y_2)^r\\
	y_1 \rightarrow y_1\\
	y_2 \rightarrow y_2\\
	y_3 \rightarrow y_3
\end{cases}.
$$

It means that $s-r = 0$, because in any other way $(\alpha \gamma)^r \gamma^{s-r} (x_1) \neq x_1$. But if $s-r = 0$, then for any $r \neq 0$ $(\alpha \gamma)^r (x_2) \neq x_2$. So, $r = 0 \Rightarrow s = 0 \Rightarrow \theta(H) \cap \ker \theta = \{id\}$.  

Here we understood that every element from $H$ can be written as $a^rc^s=(ac)^rc^{s-r}$, then we understood $ac$ fixes $x_1$, and $c^{s-r}$ fixes $x_1$ only when $s=r$, so, we moved to the element $(ac)^r$. After it we looked to all degrees of $(ac)^r$ and realized that it can go to the identical map only when $r=1$. 

Here write all the degrees of $a,b,c$.
Degrees of a:
$$
\theta(a^{3m}) =
\begin{cases}
	x_1 \rightarrow x_1 (y_3 y_2 y_1)^{-m}\\
	x_2 \rightarrow x_2 (y_1 y_3 y_2)^{-m}\\
	x_3 \rightarrow x_3 (y_3 y_1 y_2)^{2m}\\
	y_1 \rightarrow y_1\\
	y_2 \rightarrow y_2\\
	y_3 \rightarrow y_3
\end{cases}
\theta(a^{3m+1}) =
\begin{cases}
	x_1 \rightarrow x_1 y_3^{-1} (y_2 y_3 y_1)^{-m}\\
	x_2 \rightarrow x_2 y_1^{-1} (y_3 y_1 y_2)^{-m}\\
	x_3 \rightarrow x_3 y_3 y_1 (y_2 y_3 y_1)^{2m}\\
	y_1 \rightarrow y_3\\
	y_2 \rightarrow y_1\\
	y_3 \rightarrow y_2
\end{cases} 
$$ 
$$
\theta(a^{3m+2}) =
\begin{cases}
	x_1 \rightarrow x_1 y_3^{-1}y_2^{-1} (y_1 y_2 y_3)^{-m}\\
	x_2 \rightarrow x_2 y_1^{-1}y_3^{-1} (y_2 y_3 y_1)^{-m}\\
	x_3 \rightarrow x_3 y_3 (y_1 y_2y_3)^{2m+1}\\
	y_1 \rightarrow y_2\\
	y_2 \rightarrow y_3\\
	y_3 \rightarrow y_1
\end{cases}
$$
Degrees of b:
$$
\theta(b^{2k}) =
\begin{cases}
	x_1 \rightarrow x_1\\
	x_2 \rightarrow x_2 (y_2 y_3)^{-k}\\
	x_3 \rightarrow x_3 (y_3 y_2)^k\\
	y_1 \rightarrow y_1\\
	y_2 \rightarrow y_2\\
	y_3 \rightarrow y_3
\end{cases} 
\theta(b^{2k+1}) =
\begin{cases}
	x_1 \rightarrow x_1 \\
	x_2 \rightarrow x_2 y_3^{-1} (y_3 y_2)^{-k}\\
	x_3 \rightarrow x_3 y_3 (y_2 y_3)^k\\
	y_1 \rightarrow y_1\\
	y_2 \rightarrow y_3\\
	y_3 \rightarrow y_2
\end{cases}
$$
Degrees of c:
$$
\theta(c^{3k}) =
\begin{cases}
	x_1 \rightarrow x_1 (y_1 y_2y_3)^{k}\\
	x_2 \rightarrow x_2 (y_1^{-1} y_3^{-1} y_2^{-1})^{2k}\\
	x_3 \rightarrow x_3 (y3 y_1 y_2)^{k}\\
	y_1 \rightarrow y_1\\
	y_2 \rightarrow y_2\\
	y_3 \rightarrow y_3
\end{cases}
\theta(c^{3k+1}) =
\begin{cases}
	x_1 \rightarrow x_1 y_1 (y_2 y_3 y_1)^{k}\\
	x_2 \rightarrow x_2 y_1^{-1} y_3^{-1}(y_2^{-1} y_1^{-1} y_3^{-1})^{2k}\\
	x_3 \rightarrow x_3 y_3 (y_1 y_2 y_3)^k\\
	y_1 \rightarrow y_1\\
	y_2 \rightarrow y_2\\
	y_3 \rightarrow y_3
\end{cases} 
$$
$$
\theta(c^{3k+2}) =
\begin{cases}
	x_1 \rightarrow x_1 y_1 y_2 (y_3 y_1 y_2)^{k} \\
	x_2 \rightarrow x_2 y_1^{-1} y_3^{-1} y_2^{-1} y_1^{-1} (y_1^{-1} y_3^{-1} y_2^{-1})^{2k}\\
	x_3 \rightarrow x_3 y_3 y_1 (y_2 y_3 y_1)^k\\
	y_1 \rightarrow y_3\\
	y_2 \rightarrow y_1\\
	y_3 \rightarrow y_2
\end{cases}
$$
\section{Linear representation}
Introduce the linear representation we got from our representation by automorphisms.

Let build a matrix for our representation on $n=3$ strands. Get basis $V = span$ { $e_1 \dots e_n, q_1 \dots q_n$ }. Then for $\sigma_i$ 
$$
A_i = 
\begin{cases}
	e_i \rightarrow e_{i+1}q_{i+1}\\
	e_{i+1}\rightarrow  e_i-q_{i+1}\\
	q_i\rightarrow  q_i\\
	q_{i+1}\rightarrow  q_{i+1}\\
\end{cases}
$$
For $\rho_i$ 
$$
B_i = 
\begin{cases}
	e_i\rightarrow e_{i+1}\\
	e_{i+1}\rightarrow e_i\\
	q_i\rightarrow q_{i+1}\\
	q_{i+1}\rightarrow q_i\\
\end{cases}
$$

There are linear representation for $\sigma_i$ and $\sigma_{i+1}$ :
\begin{equation*}
A_1 = 
\begin{pmatrix}
  0& 1& 0& 0& 0& 0\\
  1& 0& 0& 0& 0& 0\\
  0& 0& 1& 0& 0& 0\\
  0& 0& 0& 1& 0& 0\\
  1& -1& 0& 0& 1& 0\\
  0& 0& 0& 0& 0& 1\\
\end{pmatrix}
\end{equation*}
\begin{equation*}
A_2 = 
\begin{pmatrix}
 1& 0& 0& 0& 0& 0\\
  0& 0& 1& 0& 0& 0\\
  0& 1& 0& 0& 0& 0\\
  0& 0& 0& 1& 0& 0\\
  0& 0& 0& 0& 1& 0\\
  0& 1& -1& 0& 0& 1\\
\end{pmatrix}
\end{equation*}
And there are linear representation for $\rho_i$ and $\rho_{i+1}$ :
\begin{equation*}
B_1 = 
\begin{pmatrix}
0& 1& 0& 0& 0& 0\\
  1& 0& 0& 0& 0& 0\\
  0& 0& 1& 0& 0& 0\\
  0& 0& 0& 0& 1& 0\\
  0& 0& 0& 1& 0& 0\\
  0& 0& 0& 0& 0& 1\\
\end{pmatrix}
\end{equation*}
\begin{equation*}
B_2 = 
\begin{pmatrix}
  1& 0& 0& 0& 0& 0\\
  0& 0& 1& 0& 0& 0\\
  0& 1& 0& 0& 0& 0\\
  0& 0& 0& 1& 0& 0\\
  0& 0& 0& 0& 0& 1\\
  0& 0& 0& 0& 1& 0\\
\end{pmatrix}
\end{equation*}
Note that the group $FVB_n$ is linear (it is know fact).

Is it true that the linear representation has the same kernel as representation by automorphisms?

\section{Representations of gauss virtual braids}

We recall that the group of Gauss virtual braids on $n$ strands is the group $GVB_n$ with generators  $\sigma _1,\sigma _2,\dots,\sigma _{n-1}\rho_1,\rho _2,\dots,\rho _{n-1}$ and defining relations (\ref{def}) with new relations 

\begin{align*}
\sigma_i\rho_i=\rho_i\sigma_i && i = 1, 2, \dots,n-1.
\end{align*}

We might extend representation $\theta_n$, introduced previously, to $GVB_n$ in the followin way. Let $G=\langle x_1, x_2, \ldots, x_n, y_1, y_2, \ldots, y_n, z ~|~ z^2\rangle$, then the mapping $\theta_n^*:GVB_n\to {\rm Aut}(G)$, defined by formulas below, is a representation which do not preserve the forbidden relations. 

\begin{equation}\label{defGV}
\theta_n^*(\sigma_i):
\begin{cases}
x_i \mapsto x_{i+1}z\\
x_{i+1} \mapsto x_iz\\
\end{cases}\quad
\theta_n^*(\rho_i):
\begin{cases}
x_i \mapsto x_{i+1}\\
x_{i+1} \mapsto x_i\\
y_i \mapsto y_{i+1}\\
y_{i+1} \mapsto y_i\\
\end{cases}
\end{equation}

For description of representation's kernel introduce following

\begin{definition}
Let $c$ be an element of $FVB_n$. Reduction of $\sigma$ in $b$ is element $R_{\sigma}(c)$ of $FVB_n$ derived from $c$ by deleting all $\sigma_i$ in $b$.
\end{definition}

For example, $R_{\sigma}(\rho_1\sigma_3\sigma_7\rho_5)=\rho_1\rho_5$. 

\begin{theorem}
For kernel and image of representation $\theta_n^*$ following propositions are hold.
\begin{enumerate}
\item ${\rm Im}(\theta_n^*)=S_n\times S_n.$
\item $b\in {\rm Ker}(\theta_n^*) \Leftrightarrow b\in GVP_n$ and $R_{\sigma}(b)\in VP_n.$
\end{enumerate}
\end{theorem}

\begin{proof}
\begin{enumerate}
\item Note that $GVB_n$ be generated by $\sigma_1, \ldots \sigma_{n-1}, \eta_1, \ldots \eta_{n-1}$ where $\eta_i=\sigma_i\rho_i$, since $\rho_i=e\rho_i=\sigma_i^2\rho_i=\sigma_i\eta_i$. For images of new generators we obtain:

\begin{equation}\label{newdef}
\theta_n^*(\sigma_i):
\begin{cases}
x_i \mapsto x_{i+1}z\\
x_{i+1} \mapsto x_iz\\
\end{cases}\quad
\theta_n^*(\eta_i):
\begin{cases}
x_i \mapsto x_{i}z\\
x_{i+1} \mapsto x_{i+1}z\\
y_i \mapsto y_{i+1}\\
y_{i+1} \mapsto y_i\\
\end{cases}
\end{equation}

Every automorphism $A$ lying in $Im(\theta_n^*)$ has the following form:

\begin{align*}
A:
\begin{cases}
x_i\mapsto x_{s(i)}z^{\varepsilon_i}\\
y_j\mapsto y_{t(j)}
\end{cases}
\end{align*}

Here $s,t\in S_n$, $\varepsilon_i \in \{0,1\}$ and $s,t$ define $\varepsilon_i$. 

\item Let $b\in {\rm Ker}(\theta_n^*)$, then $b$ actes on $x_i$ identically for all $i$, this means that $b\in GVP_n$. Also $b$ actes on $y_i$ identically for all $i$, then $R_{\sigma}(b)\in VP_n$, since all $\sigma_i$ not actes on every generator $y_j$.

Now, let $b\in GVP_n$ and $R_{\sigma}(b)\in VP_n$. Note that $b$ actes on every $y_i$ identically, this follows from the second condition. Since $b$ is pure braid, we obtain that $x_i\xrightarrow{b}x_iz^{\varepsilon_i}$ where $\varepsilon_i\in \{0, 1\}$. Not hard to see that total exponent sum of $\sigma_i$ in $b$ is even number. Also total exponent sum of $\eta_i$ in $b$ is even number. Then $\varepsilon_i=0$ for all $i$. This means that $b\in {\rm Ker}(\theta_n^*)$.
\end{enumerate}
\end{proof}

\section{Knot theory}
Can we apply it to knot theory?

Explain why cannot we directly apply it to the knot theory.

Can we do something exotic to apply it somehow to knot theory?

\section{Questions}
Is the closure of the braid from the kernel trivial?

Look to the closure of $(\rho_1\sigma_2)^3$. Is the closure of this braid trivial?

Look to the other representations. Are they representations? Does our element belong to the kernel?

Other representations.

Representations without any additional conditions:

\begin{equation}\label{defGV}
\theta_n^*(\sigma_i):
\begin{cases}
x_i \mapsto x_{i+1}y_{i+1}^m\\
x_{i+1} \mapsto x_iy_{i+1}^{-m}\\
\end{cases}\quad
\theta_n^*(\rho_i):
\begin{cases}
x_i \mapsto x_{i+1}\\
x_{i+1} \mapsto x_i\\
y_i \mapsto y_{i+1}\\
y_{i+1} \mapsto y_i\\
\end{cases}
\end{equation}

\begin{equation}\label{defGV}
\theta_n^*(\sigma_i):
\begin{cases}
x_i \mapsto x_{i+1}y_{i+1}^z\\
x_{i+1} \mapsto x_i(y_{i+1}^{-1})^z\\
\end{cases}\quad
\theta_n^*(\rho_i):
\begin{cases}
x_i \mapsto x_{i+1}\\
x_{i+1} \mapsto x_i\\
y_i \mapsto y_{i+1}\\
y_{i+1} \mapsto y_i\\
\end{cases}
\end{equation}

If $y_i z = z y_i$ for all i, next maps are representations:

\begin{equation}\label{defGV}
\theta_n^*(\sigma_i):
\begin{cases}
x_i \mapsto x_{i+1}y_{i+1}\\
x_{i+1} \mapsto x_iy_{i+1}^{-1}\\
\end{cases}\quad
\theta_n^*(\rho_i):
\begin{cases}
x_i \mapsto x_{i+1}z\\
x_{i+1} \mapsto x_iz^{-1}\\
y_i \mapsto y_{i+1}\\
y_{i+1} \mapsto y_i\\
\end{cases}
\end{equation}

\begin{equation}\label{defGV}
\theta_n^*(\sigma_i):
\begin{cases}
x_i \mapsto x_{i+1}y_{i+1}\\
x_{i+1} \mapsto x_iy_{i+1}^{-1}\\
\end{cases}\quad
\theta_n^*(\rho_i):
\begin{cases}
x_i \mapsto x_{i+1}^z\\
x_{i+1} \mapsto x_i^{z^{-1}}\\
y_i \mapsto y_{i+1}\\
y_{i+1} \mapsto y_i\\
\end{cases}
\end{equation}

Lets define following elements
\begin{align*}
&\lambda_{i, i+1} = \rho_i\sigma_i^{-1}, \quad \lambda_{i+1,i}=\rho_i\lambda_{i,i+1}\rho_i = \sigma_i^{-1}\rho_i, \quad &&for ~ i = 1,2,\dots,n-1,\\
&\lambda_{i,j} = \rho_{j-1}\rho_{j-2}\dots\rho_{i+1}\lambda_{i,i+1}\rho_{i+1}\dots\rho_{j-2}\rho_{j-1},\\
&\lambda_{j,i} = \rho_{j-1}\rho_{j-2}\dots\rho_{i+1}\lambda_{i,i+1}\rho_{i+1}\dots\rho_{j-2}\rho_{j-1}, \quad &&for ~ 1 \le i < j-1 \le n-1.
\end{align*}

And formulate a theorem which was rpoved by V.~Bardakov in 
The virtual and universal braids (2004).

\begin{theorem}
	The group $VP_n$ admits a presentation with the generators $\lambda_{k,l}, 1 \le k \not= l \le n$, and the defining relations:
	\begin{align*}
	\lambda_{i,j}\lambda_{k,l} &= \lambda_{k,l}\lambda_{i,j}; \\
	\lambda_{k,i}\lambda_{k,j}\lambda_{i,j} &= \lambda_{i,j}\lambda_{k,j}\lambda_{k,i}, \\
	\end{align*}
	where distinct letters stand for distinct indices.
\end{theorem}

Let us denote by $H_n$ the normal closure of $B_n$ in $VB_n$.

Define the followin elements: 
\begin{align*}
x_{i,i+1}&=\sigma_i, ~~& x_{i,j}&=\rho_{j-1}\ldots \rho_{i+1}\sigma_i\rho_{i+1}\ldots \rho_{j-1} ~{\rm for}~ 1\le i<j-1\le n-1,\\
x_{i+1,i}&=\rho_i\sigma_i\rho_i, ~~& x_{j,i}&=\rho_{j-1}\ldots \rho_{i+1}\rho_i\sigma_i\rho_i\rho_{i+1}\ldots \rho_{j-1} ~{\rm for}~ 1\le i<j-1\le n-1.\\
\end{align*}


\begin{lemma}
	Let $\rho \in S_n$. Then element $\rho x_{i,j}\rho^{-1}$ is equal to $x_{\rho(i),\rho(j)}$ for $1\le i\ne j\le n-1$.
\end{lemma}


\begin{lemma}
	The group $H_n$ admits a presentation with the generators $x_{k,l}$, $1\le k\ne l\le n$, and the defining relations:
	\begin{align*}
	x_{i,j}x_{k,l}&=x_{k,l}x_{i,j},\\
	x_{i,k}x_{k,j}x_{i,k}&=x_{k,j}x_{i,k}x_{k,j},\\
	\end{align*}
	where distinct letters stand for disting indices.
\end{lemma}

Most likely (up to my opinion) the map in the second line on the left is the representation. And this representation is better than the previous one (need to check).

Denote by $H_1$ the subgroup of $FVP_3=\langle a,b,c~|~[a,c]=1\rangle$ generated by the elements $a^3,b^2,c^3$. Find the intersection of $ker{\theta}$ with $H_1$. Is it trivial?

Denote by $H_2$ the subgroup of $FVP_3=\langle a,b,c~|~[a,c]=1\rangle$ generated by the elements $ac,b^2,c^3$. Find the intersection of $ker{\theta}$ with $H_2$. Is it trivial?

Denote by $H_3$ the subgroup of $FVP_3=\langle a,b,c~|~[a,c]=1\rangle$ generated by the elements $ac,b^2,a^3$. Find the intersection of $ker{\theta}$ with $H_3$. Is it trivial?

I choose $H_1,H_2,H_3$ in this way since the images of the elements which generate $H_1,H_2,H_3$ fix the elements $y_1,y_2,y_3$. 

Let $\alpha=(\rho_1\sigma_2)^6$, and let $x$ be an element from $VB_3$ (for example, $x=\rho_2$). Let $A$ be a subgroup in $VP_3$ generated by by two elements $\alpha$, $\beta=x^{-1}\alpha x$. Is it clear that this subgroup belongs to $ker(\theta)$. What can we say about this subgroup? Is it free? Can we compare it with the whole group $ker(\theta)$?

Can one reduce the dimension of the linear representation we got? I believe that it is possible and not difficult. The image of our linear representation acts on the vector space $V$ with the basis $e_1,e_2,e_3,q_1,q_2,q_3$. Look to the subspace $U$ generated by $e_1+e_2+e_3$, $q_1+q_2+q_3$. This subspace is invariant under $\theta(FVB_n)$. Look to the representation induced by $\theta$, which maps like $FVB_n\to GL(V/U)$.

Look to the kernel thinking about generators $\sigma_1, \sigma_2, \rho_1\sigma_1,\rho_2\sigma_2$. The first two generators move only the elements $x_1,x_2,x_3$, the second two generators move only the elements $y_1,y_2,y_3$.
\bibliographystyle{alpha}
\begin{thebibliography}{00}
\bibitem{problems}
R.~Fenn, D.~Ilyutko, L.~Kauffman, V.~Manturov, Unsolved problems in virtual knot theory and combinatorial knot theory, ArXiv:math/1409.2823.
\bibitem{BarBelDom}
V.~Bardakov, P.~Bellingeri, C.~Damiani, Unrestricted virtual braids, fused links and other quotients of virtual braid groups, J. Knot Theory Ramifications, V.~24, N.~12, 2015, 1550063.
\end{thebibliography}

Proposition 2.1. $B_n$ has a presentation with generators {$a_{ts};n \geqslant t >s \geqslant 1$} and with defining relations
\begin{eqnarray}
a_{ts}a_{rq}=a_{rq}a_{ts} \\ a_{ts}a_{sr}=a_{tr}a_{ts}=a_{sr}a_{tr}
\end{eqnarray}
Remark 2.2. Relation ~(\ref{ne znau}) asserts that $a_{ts}$ and $a_{rq}$ commute if $t$ and $s$ do
not separate $r$ and $q$. Relation ~(\ref{ne znau}) expresses a type of ``partial'' commutativity in the case when $a_{ts}$, and $a_{rq}$ share a common strand. It tells us that if the product atsasr occurs in a braid word, then we may move $a_{ts}$ to the right (resp. move $a_{sr}$ to the left) at the expense of increasing the first subscript of $a_{sr}$ to $t$ (resp. decreasing the second subscript of $a_{ts}$ to $r$.)

\end{document}
